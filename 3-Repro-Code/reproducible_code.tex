\documentclass[12pt, compress]{beamer} % ,t or ,b for top or bottom-aligned, compress for h-dots
\usepackage{graphicx}
\usepackage[absolute,overlay]{textpos} % to get full size picture
\usepackage{tikz}
\usepackage{csquotes}
\usepackage{hyperref}

%%%%% BEAMER OPTIONS 

\usetheme{default}
\setbeameroption{hide notes} % change to show to show notes
\setbeamertemplate{note page}[plain]
\addtobeamertemplate{note page}{\setlength{\parskip}{12pt}} % adds spaces to notes page

\setbeamertemplate{frametitle}[default][center] % centering titles
\setbeamertemplate{footline} % footnote with sections
{
	\begin{beamercolorbox}[center]{section in head/foot}
	\vskip12pt % distance from top 
	\insertnavigation{\textwidth}
	\vskip4pt % distance from bottom
	\end{beamercolorbox}
}

% \setbeamertemplate{footline}{% add footnotes
   % \raisebox{5pt}{\makebox[\paperwidth]{\hfill\makebox[20pt]{\color{gray}
         % \scriptsize\insertframenumber}}}\hspace*{5pt}}

\beamertemplatenavigationsymbolsempty % gets rid of navigation bar

\addtobeamertemplate{frametitle}{\vspace*{.6cm}}{\vspace*{0.2cm}} % slide title spacing

% \beamerdefaultoverlayspecification{<+->} % uncovers everything in a step-wise fashion


%%%%% SECTION TITLE SLIDES

\AtBeginSection[]{
  \begin{frame}
  \vfill
  \centering
  \begin{beamercolorbox}[center]{title}
  	\vskip14pt
    \usebeamerfont{title}\insertsectionhead\par%
  \end{beamercolorbox}
  \vfill
  \end{frame}
}

%%%%% COLORS 

\definecolor{textcolor}{RGB}{24,24,24} % sets main text color
\definecolor{background}{RGB}{255,255,255} % sets background to white
%\definecolor{background}{RGB}{24,24,24} % sets background to black
%\definecolor{title}{RGB}{107,174,214}
%\definecolor{subtitle}{RGB}{102,255,204}
\definecolor{hilight}{RGB}{102,255,204}
\definecolor{vhilight}{RGB}{255,111,207}
\definecolor{medblue}{rgb}{.1,.35,1}
\definecolor{medgreen}{RGB}{22, 158, 58}
\definecolor{burntorange}{rgb}{0.8, 0.33, 0.0}
\definecolor{ceruleanblue}{rgb}{0.16, 0.32, 0.75}
\definecolor{darkblue}{RGB}{0,0,120}
\definecolor{gray}{RGB}{155,155,155}
\definecolor{lightgrey}{rgb}{.95,.95,.95}

\setbeamercolor{titlelike}{fg=ceruleanblue}
\setbeamercolor{subtitle}{fg=burntorange}
\setbeamercolor{framesubtitle}{fg=darkblue}
\setbeamercolor{institute}{fg=gray}
\setbeamercolor{normal text}{fg=textcolor, bg=background} % sets main background and font color
\setbeamercolor{item}{fg=burntorange}
\setbeamercolor{enumerate}{fg=burntorange}
\hypersetup{colorlinks, linkcolor=, urlcolor=burntorange} 
\let\oldtexttt\texttt% Store \texttt
\renewcommand{\texttt}[2][ceruleanblue]{\textcolor{#1}{\ttfamily #2}}% \texttt[<color>]{<stuff>}

%%%%% FONTS 
\usepackage[english]{babel}
\usefonttheme{professionalfonts}
\usefonttheme{serif}
\usepackage{fontspec}
\setmainfont[Ligatures=TeX]{Helvetica Light} % ligatures for em dash and quotation marks
\setbeamerfont{note page}{family*=pplx,size=\footnotesize} % Palatino for notes
\setbeamerfont{framesubtitle}{size=\small} 

%%%%% SPACING

\let\olditem\item % increase spacing between bullets
\renewcommand{\item}{%
\olditem\vspace{\fill}}
\setlength{\tabcolsep}{0.5em} % horizontal padding for tables 


%%%%% MACROS

\newcommand{\bi}{\begin{itemize}}
\newcommand{\ei}{\end{itemize}}
\newcommand{\ig}{\includegraphics}
\newcommand{\subt}[1]{{\footnotesize \color{burntorange} {#1}}}
\newcommand{\nb}[1]{{\color{burntorange} {#1}}}

%%%%% TITLE MATERIAL %%%%

\title[Short Title]{Coding Reproducibility}
\subtitle{Lessons Learned \vspace{-20pt} }
\institute{\vspace{10pt} \includegraphics[width = 40mm]{images/pdel.png}\vspace{10pt} \includegraphics[width = 20mm]{images/UCSDlogo}}
\date[Short Occasion]{Julia Clark \\ BITSS Transparency and Reproducibility Workshop \\ New Delhi \\ 16 March 2017}

%%%%%%%%%%%%%%%%%%% BEGIN %%%%%%%%%%%%%%%%%%
\begin{document}


%%%%% TITLE PAGE

{\setbeamertemplate{footline}{} % no navigation
\frame{
  \titlepage
  \note{}
  }
}


%%%%%%%%%%%%%%%%% OVERVIEW %%%%%%%%%%%%%%%%


	{\setbeamertemplate{footline}{} % no navigation
	\begin{frame}{Overview}
		\begin{enumerate}
			\item About PDEL data transparency project
			\item What makes code reproducible?
			\item \textbf{Lessons Learned}:
		\end{enumerate}
	\end{frame}
	}

%%%%%%%%%%%%%%%%% PDEL %%%%%%%%%%%%%%%%

\section{About PDEL}
\subsection{pdel}
	\begin{frame}{UCSD's Policy Design and Evaluation Lab (pdel.ucsd.edu)}
		 Rigorous social science methodology \textcolor{ceruleanblue}{+} information technology \textcolor{ceruleanblue}{$\rightarrow$} policies and programs that alleviate poverty, promote health, welfare, and security, and enhance accountability.
		 \begin{itemize}
		 	\item Interdisciplinary collaboration
		 	\item Project and fund management support 
		 	\item \textcolor{burntorange}{Data transparency services---helping researchers replicate studies and prepare files for public dissemination}
		 \end{itemize}
\end{frame}

%%%%%%%%%%%%%%%%%%%% WHAT IT IS %%%%%%%%%%%%%%%%%%%%%%%

\section{What Makes Code Reproducible?}

\subsection{What}
	\begin{frame}{Replication files that are \dots}
		\begin{enumerate}
			\item Complete but parsimonious
			\item Run and reproduce results with one click
			\item Readable and interpretable by humans
			\item Protect personal information
		\end{enumerate}
	\end{frame}

\subsection{Why}		
	\begin{frame}{Why do we care?}
		\begin{itemize}
			\item \textbf{Unselfish reasons}---part of the scientific process and a public good
			\item \textbf{Selfish reasons}---make code more usable for yourself, catch potentially embarrassing errors before they become public, boost your transparency credibility
		\end{itemize}
		
	\end{frame}
	
%%%%%%%%%%%%%%%%% COMPLETE %%%%%%%%%%%%%%%%%

\section{Lessons Learned}
\subsection{Complete}
	\begin{frame}{1. Complete and Parsimonious}
	
	\textbf{Complete:} All materials needed to generate and decipher results are included in the replication files \dots

		\begin{itemize}		
			\item \textcolor{burntorange}{Code}---for analysis AND cleaning/merging data files
			\item \textcolor{burntorange}{Data}---raw and manipulated
			\item \textcolor{burntorange}{Supplementary files}---codebooks, readme files, etc.
		\end{itemize} 
		
		\bigskip
		
		
	\textbf{Parsimonious:} Unnecessary files (e.g., old versions of figures, tables, data not used in analysis) should NOT be included---\textcolor{burntorange}{AKA, don't just share your project directory as-is!}
			
	\end{frame}


%%%%%%%%%%%%%%%%% RUNS %%%%%%%%%%%%%%%%%%%%
\subsection{Runs}
	\begin{frame}{2. Runs \& Reproduces}
		Code and data should \textbf{reproduce} the paper's results without error. 
		
		
		\begin{itemize}
			\item This includes ALL tables, graphs, etc. in paper
			\item Ideally code can be executed with a \textbf{single click}
			\item Great if it runs on your machine, but always good to test on other computer/OS/software version to debug
		\end{itemize}
		
			
	\end{frame}

%%%%%%%%%%%%%%%%% READABLE %%%%%%%%%%%%%%%%%
\subsection{Readable}

	\begin{frame}{3. Readable and interpretable (by humans!)}
		Code should be streamlined and legible, with intuitively organized files. 
		
		
		\begin{itemize}		
			\item Clearly labeled files within a logical folder structure
			\item Separate code for data analysis/merging/cleaning, ideally with master script to run all
			\item Comment to help reader navigate/interpret
			\item Declutter syntax (ample use of spaces, indentation, headers)
			\item Code for main analysis should be prominent \& clearly labeled
		\end{itemize} 
			
	\end{frame}

%%%%%%%%%%%%%%%%% PROTECTS PII %%%%%%%%%%%%%%%
\subsection{Protecting PII}
 
	\begin{frame}{4. Protects PII}
		 \textbf{Personally identifiable information (PII)} includes \textcolor{burntorange}{direct identifiers} (e.g., name, address, etc.) and \textcolor{burntorange}{indirect identifiers}---personal characteristics that, when combined, can identify a person.
			 
			 
		\begin{itemize}		
			\item This info should be removed/altered in public data files, with original files stored securely
			\item When possible, anonymize \textit{before} merging/cleaning so that data and code for these processes can be shared publicly
		\end{itemize} 
		\bigskip
			
	\end{frame}
 
\begin{frame}{}
	\bigskip
	\centering \large Questions?
\end{frame}
























\end{document}